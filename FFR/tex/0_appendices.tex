\section{\gnuplot{} Quellcode zur Auswertung von Teilversuch 1}
    \subsection{Senkrechter Fall}
    \label{appdx:tv1-senk-gp}
    {  
        % % Surpress "errors" in minted
        % https://tex.stackexchange.com/a/289068
        \renewcommand{\fcolorbox}[4][]{#4}
        \begin{minted}[linenos,breaklines,autogobble,frame=leftline,framesep=10pt]{gnuplot}
#!/usr/bin/env gnuplot

set term epslatex color size 6in, 4in
set output "tv1-senkrecht.tex"
set decimalsign locale 'de_DE.UTF-8'

set title "Reflektionskoeffizient gegen Einfallswinkel"
set ylabel "Reflektionskoeffizient $\\zeta^\\bot$ (Einheitslos)"
set xlabel "Einfallswinkel $\\alpha$ ($\\si{\\degree}$)"

set mxtics
set mytics
set samples 10000

f(x) = ((sqrt(n**2 - (sin(pi*x/180))**2) - cos(pi*x/180))**2)/(n**2 - 1)

n = 1.55
# (x, y, xdelta, ydelta)
fit f(x) "senkrecht.dat" u 1:2:(2.5):3 xyerrors via n

# Linien
set key bottom right spacing 2

titel = "$\\frac{\\left(\\sqrt{(".gprintf("%.5f", n).")^2 - \\sin^2\\alpha} - \\cos\\alpha\\right)^2}{(".gprintf("%.5f", n).")^2 - 1}$"
plot f(x) title titel lc rgb 'dark-magenta', \
    "senkrecht.dat" u 1:2:(2.5):3 with xyerrorbars title "Messpunkte" pointtype 0 lc rgb 'dark-goldenrod'
        \end{minted}
    }
    mit \texttt{senkrecht.dat}:
    \begin{multicols}{2}
        \begin{minted}[linenos,breaklines,autogobble,frame=leftline,framesep=10pt]{text}
#alpha/deg  Zeta    deltaZeta
5           0,191   0,004
10          0,195   0,005
15          0,200   0,005
20          0,208   0,005
25          0,222   0,005
30          0,224   0,005
35          0,243   0,006
40          0,260   0,006
45          0,279   0,006
50          0,306   0,007
55          0,347   0,008
60          0,378   0,008
65          0,421   0,009
70          0,456   0,010
75          0,554   0,012
80          0,595   0,013
        \end{minted}
    \end{multicols}
    \vspace{-\baselineskip}
    Rohausgabe:
    \begin{minted}[linenos,breaklines,autogobble,frame=leftline,framesep=10pt]{text}
After 3 iterations the fit converged.
final sum of squares of residuals : 16.0509
rel. change during last iteration : -1.47495e-06

degrees of freedom    (FIT_NDF)                        : 15
rms of residuals      (FIT_STDFIT) = sqrt(WSSR/ndf)    : 1.03444
variance of residuals (reduced chisquare) = WSSR/ndf   : 1.07006
p-value of the Chisq distribution (FIT_P)              : 0.378676

Final set of parameters            Asymptotic Standard Error
=======================            ==========================
n               = 1.46469          +/- 0.00659      (0.4499%)
    \end{minted}

    \subsection{Paralleler Fall}
    \label{appdx:tv1-parl-gp}
    {  
        % % Surpress "errors" in minted
        % https://tex.stackexchange.com/a/289068
        \renewcommand{\fcolorbox}[4][]{#4}
        \begin{minted}[linenos,breaklines,autogobble,frame=leftline,framesep=10pt]{gnuplot}
#!/usr/bin/env gnuplot

set term epslatex color size 6in, 4in
set output "tv1-parallel.tex"
set decimalsign locale 'de_DE.UTF-8'

set title "Reflektionskoeffizient$^2$ gegen Einfallswinkel"
set ylabel "Reflektionskoeffizient$^2$ $\\left(\\zeta^\\parallel\\right)^2$ (Einheitslos)"
set xlabel "Einfallswinkel $\\alpha$ ($\\si{\\degree}$)"

set mxtics
set mytics
set samples 10000

f(x) = ((n*n*cos(x*pi/180)-sqrt(n*n - sin(x*pi/180)*sin(x*pi/180))) / (n*n*cos(x*pi/180)+sqrt(n*n - sin(x*pi/180)*sin(x*pi/180))))**2

n = 1.55
# (x, y, xdelta, ydelta)
fit f(x) "parallel.dat" u 1:2:(2.5):3 xyerrors via n

# Linien
set key top right spacing 2

titel = "$\\left(\\frac{(".gprintf("%.5f", n).")^2\\cos(\\alpha) - \\sqrt{(".gprintf("%.5f", n).")^2-\\sin^2(\\alpha)}}{(".gprintf("%.5f", n).")^2\\cos(\\alpha) + \\sqrt{(".gprintf("%.5f", n).")^2-\\sin^2(\\alpha)}}\\right)^2$"
plot f(x) title titel lc rgb 'dark-magenta', \
    "parallel.dat" u 1:2:(2.5):3 with xyerrorbars title "Messpunkte" pointtype 0 lc rgb 'dark-goldenrod'
        \end{minted}
    }
    mit \texttt{parallel.dat}:
    \begin{multicols}{2}
        \begin{minted}[linenos,breaklines,autogobble,frame=leftline,framesep=10pt]{text}
# alpha/deg tilde{I} delta tilde{I}
5           0,0416   0,0017
10          0,0409   0,0016
15          0,0393   0,0016
20          0,0361   0,0015
25          0,0321   0,0013
30          0,0295   0,0012
35          0,0226   0,0009
40          0,0188   0,0008
45          0,0127   0,0005
50          0,00739  0,00029
52,5        0,00545  0,00022
55          0,00289  0,00012
57,5        0,00164  0,00007
60          0,00187  0,00008
62,5        0,00348  0,00014
65          0,00746  0,00029
67,5        0,0144   0,0006
70          0,0311   0,0013
75          0,076    0,003
80          0,127    0,005
        \end{minted}
    \end{multicols}
    \vspace{-\baselineskip}
    Rohausgabe:
    \begin{minted}[linenos,breaklines,autogobble,frame=leftline,framesep=10pt]{text}
After 3 iterations the fit converged.
final sum of squares of residuals : 30.1409
rel. change during last iteration : -8.25091e-16

degrees of freedom    (FIT_NDF)                        : 19
rms of residuals      (FIT_STDFIT) = sqrt(WSSR/ndf)    : 1.25951
variance of residuals (reduced chisquare) = WSSR/ndf   : 1.58636
p-value of the Chisq distribution (FIT_P)              : 0.0500322

Final set of parameters            Asymptotic Standard Error
=======================            ==========================
n               = 1.52605          +/- 0.008781     (0.5754%)
    \end{minted}

\section{\gnuplot{} Quellcode zur Auswertung von Teilversuch 2}
    \label{appdx:tv2-gnuplot}
    {  
        % % Surpress "errors" in minted
        % https://tex.stackexchange.com/a/289068
        \renewcommand{\fcolorbox}[4][]{#4}
        \begin{minted}[linenos,breaklines,autogobble,frame=leftline,framesep=10pt]{gnuplot}
#!/usr/bin/env gnuplot

set term epslatex color size 6in, 4in
set output "tv2-plot.tex"
set decimalsign locale 'de_DE.UTF-8'

set title "Drehwinkel der Polarisationsebene gegen Einfallswinkel"
set ylabel "Drehwinkel $\\Psi$ ($\\si{\\degree}$)"
set xlabel "Einfallswinkel $\\alpha$ ($\\si{\\degree}$)"

set mxtics
set mytics
set samples 10000

f(x) = (180/pi)*atan(-(cos(x*pi/180)*sqrt(n*n - sin(x*pi/180)*sin(x*pi/180)))/(sin(x*pi/180)*sin(x*pi/180)))

n = 1.55
# (x, y, xdelta, ydelta)
fit f(x) "tv2.dat" u 1:2:(2.5):(0.8) xyerrors via n

# Linien
set key bottom right spacing 2

titel = "$\\arctan \\left(- \\frac{\\cos\\alpha \\sqrt{(".gprintf("%.5f", n).")^2-\\sin^2\\alpha}}{\\sin^2\\alpha} \\right)$"
plot f(x) title titel lc rgb 'dark-magenta', \
    "tv2.dat" u 1:2:(2.5):(0.8) with xyerrorbars title "Messpunkte" pointtype 0 lc rgb 'dark-goldenrod'
        \end{minted}
    }
    mit \texttt{tv2.dat}:
    \begin{multicols}{2}
        \begin{minted}[linenos,breaklines,autogobble,frame=leftline,framesep=10pt]{text}
# alpha/deg Psi/deg
15,0        -89,0
20,0        -87,0
25,0        -84,5
30,0        -81,0
35,0        -76,5
40,0        -73,0
45,0        -67,0
50,0        -60,0
52,5        -56,5
55,0        -52,0
57,5        -48,0
60,0        -43,0
65,0        -35,0
70,0        -25,0
75,0        -14,5
80,0        -11,0
        \end{minted}
    \end{multicols}
    \vspace{-\baselineskip}
    Rohausgabe:
    \begin{minted}[linenos,breaklines,autogobble,frame=leftline,framesep=10pt]{text}
After 5 iterations the fit converged.
final sum of squares of residuals : 9.14718
rel. change during last iteration : -1.04167e-06

degrees of freedom    (FIT_NDF)                        : 15
rms of residuals      (FIT_STDFIT) = sqrt(WSSR/ndf)    : 0.780905
variance of residuals (reduced chisquare) = WSSR/ndf   : 0.609812
p-value of the Chisq distribution (FIT_P)              : 0.869697

Final set of parameters            Asymptotic Standard Error
=======================            ==========================
n               = 1.69105          +/- 0.04292      (2.538%)
    \end{minted}